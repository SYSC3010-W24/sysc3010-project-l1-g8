\section{Reflections}

\subsection{Project Success}

Overall, the team considers our project, Fire Alarm Notification System (FANS), a success. We were able to develop a
comprehensive real-time system that addresses the issues that arise with preventable fire-related deaths in Canada.
FANS uses modern web technologies (Flask) and communication protocols (I2C, UDP, and HTTP) to enhance its
effectiveness.

FANS functions as we intended: detecting smoke and high temperatures, triggering multiple alarms, and sending prompt
notifications to registered users. The system's ability to detect potential fires and alert users in real-time is a
major part of the project’s success.

FANS sensors were accurate and reliable, ensuring possible fire hazards are detected rapidly. The system can easily be
expanded by adding more alarm units and accessory devices that run similar code to FANS’s current implementation. Our
GUI provides an intuitive monitoring and control interface for users, allowing one to check the emergency status,
review sensor data, and manage alarms.

The FANS project can be considered a success due to its functionality, reliability, and scalability.

\subsection{Learning Experiences}

Developing FANS gave our team valuable experience and insight into creating a multi-faceted product.

One of the most significant learning experiences was gaining hands-on experience with the Raspberry Pi Ecosystem. Getting 
using the Raspberry Pi's terminal, learning micropython for the Pico W, and applying our Electronics knowledge to build 
circuits, were all essential to the growth of our technical ability.

Our time-management skills were strengthened as we learned to juggle report writing and coding, while doing other 
courses. Communication amongst the team was critical, and we learned to balance our varying skill sets with different 
aspects of the project. We resolved conflicts (scheduling and opinion-based) constructively while working together to 
meet all our milestones.

\subsection{Project Extensions}

The first project extension that could be made is the inclusion of geographically aware system nodes. FANS could make 
system nodes geographically aware so that the emergency response would only target affected users. In addition, 
haptic alarm buzzers should only buzz if the wearer is within a certain range of the affected area. Only users who would 
be within the affected area should be notified via email and SMS to steer clear of the area. Using geographic location 
information, FANS could be modified to subscribe users to emergency response measures only if they are within the affected area.
