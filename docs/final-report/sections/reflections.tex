\section{Reflections}

\subsection{Project Success}

Overall, the team considers our FANS to be a success. We were able to develop a comprehensive real-time system that
addresses the issues that arise with preventable fire-related deaths in Canada. FANS uses modern web technologies and
multiple communication protocols (I2C, UDP, and HTTP) to enhance its effectiveness.

FANS functions as we intended: detecting smoke and high temperatures, triggering multiple alarms, and sending prompt
notifications to registered users. The system's ability to detect potential fires and alert users in real-time is a
major part of the project’s success.

FANS sensors were reliable during testing. The system is easily expanded by simply running each distributed node on
more devices, which is a key use case for installing the system in larger buildings. Our GUI provides an intuitive
monitoring and control interface for users, allowing the user to check the emergency status, review the system's
environmental conditions, and manage alarm timeouts.

The FANS project can be considered a success due to its functionality, reliability, and scalability. It meets all of
the functional requirements originally laid out in the project proposal.

\subsection{Learning Experiences}

Developing FANS gave our team valuable experience and insight into creating a multi-faceted product.

One of the most significant learning experiences was gaining hands-on experience with the Raspberry Pi ecosystem.
Getting using the Raspberry Pi's terminal, learning MicroPython for the Pico, and applying our electronics knowledge to
build circuits connected to microcontrollers were all essential to the growth of our technical ability.

Our time-management skills were also strengthened, as we learned to write our reports alongside development to keep
records of our progress and design evolution. Communication amongst the team was critical, and we learned to leverage
our varying skill sets across different aspects of the project. We resolved conflicts (scheduling and opinion-based)
constructively while working together to meet all our milestones.

\subsection{Project Extensions}

The inclusion of geographically aware system nodes would make FANS a more robust system. Making FANS nodes
geographically aware would allow the emergency response to only target affected users. In addition, haptic alarm
buzzers would only buzz if the wearer is within a certain range of the affected area. Only users within the affected
area should be notified via email and SMS to steer clear of the area. Using geographic location information, FANS could
be modified to subscribe users to emergency response measures only if they are within the affected area, making it more
effective.
