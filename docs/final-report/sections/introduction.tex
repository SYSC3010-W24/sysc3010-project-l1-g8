\section{Project Description}

\subsection{Motivation}

The motivation for the FANS project was to address the critical problem of preventable fire-related deaths in Canada.
Fire safety was a fundamental concern for individuals, families, and communities nationwide. By leveraging modern
technological advancements, the FANS system aimed to significantly enhance the effectiveness of fire alarm systems,
reducing response times, and ultimately saving lives. The importance of this project could not be overstated, as it
directly impacted the safety and well-being of Canadians.

\subsection{Problem Statement}

The need for the Fire Alarm Notification System (FANS) stemmed from the alarming number of preventable fire deaths in
Canada, where 220 people died in fires each year, with at least one in seven of these deaths occurring in homes without
working smoke alarms \cite{fire-stats}. This critical problem underscored the shortcomings of traditional fire alarm
systems, which often lacked advanced communication capabilities and real-time monitoring capabilities
\cite{modern-fire-alarms}. These limitations not only contributed to delayed response times but also to preventable
deaths, underscoring the urgent need for an advanced fire alarm solution.

Traditional systems’ shortcomings, coupled with the fact that smoke detection systems were sometimes unsafely disarmed
by users to avoid false alarms—especially those installed close to kitchen spaces—further exacerbated the problem. The
FANS project sought to address these issues by integrating smoke and temperature sensors with Internet of Things (IoT)
technology. This approach not only aimed to cover scenarios where smoke may not reach the detecting device but also
offered real-time notifications via SMS and email, thus notifying homeowners immediately in the event of an emergency.

By providing configurable thresholds for smoke and temperature alarms and adjustable timeouts that prevented the system
from being deactivated in an unsafe manner, FANS aimed to use technological advances to significantly improve the
effectiveness of fire detection systems. The ultimate goal of the project was to reduce response times, improve overall
fire safety, and thereby mitigate fire disasters and protect the well-being of individuals, families, and communities
across Canada \cite{smoke-alarm-gc}.

\subsection{Overview of Solution}
