\section{Project Description}

\subsection{Motivation}

The motivation for the FANS project was to address the critical problem of preventable fire-related deaths in Canada.
Fire safety was a fundamental concern for individuals, families, and communities nationwide. By leveraging modern
technological advancements, the FANS system aimed to significantly enhance the effectiveness of fire alarm systems,
reducing response times, and ultimately saving lives. The importance of this project could not be overstated, as it
directly impacted the safety and well-being of Canadians.

\subsection{Problem Statement}

The need for the Fire Alarm Notification System (FANS) stemmed from the alarming number of preventable fire deaths in
Canada, where 220 people died in fires each year, with at least one in seven of these deaths occurring in homes without
working smoke alarms \cite{fire-stats}. This critical problem underscored the shortcomings of traditional fire alarm
systems, which often lacked advanced communication capabilities and real-time monitoring capabilities
\cite{modern-fire-alarms}. These limitations not only contributed to delayed response times but also to preventable
deaths, underscoring the urgent need for an advanced fire alarm solution.

Traditional systems’ shortcomings, coupled with the fact that smoke detection systems were sometimes unsafely disarmed
by users to avoid false alarms—especially those installed close to kitchen spaces—further exacerbated the problem. The
FANS project sought to address these issues by integrating smoke and temperature sensors with Internet of Things (IoT)
technology. This approach not only aimed to cover scenarios where smoke may not reach the detecting device but also
offered real-time notifications via SMS and email, thus notifying homeowners immediately in the event of an emergency.

By providing configurable thresholds for smoke and temperature alarms and adjustable timeouts that prevented the system
from being deactivated in an unsafe manner, FANS aimed to use technological advances to significantly improve the
effectiveness of fire detection systems. The ultimate goal of the project was to reduce response times, improve overall
fire safety, and thereby mitigate fire disasters and protect the well-being of individuals, families, and communities
across Canada \cite{smoke-alarm-gc}.

\subsection{Overview of Solution}

In our project, we implemented the Fire Alarm Notification System (FANS), a real-time system designed for efficient
fire detection and alerting. The system comprised several key components, seamlessly integrated to ensure a
comprehensive approach to fire safety. We outlined the system's architecture in a detailed deployment diagram,
illustrating the interconnections between the various components.

The Alarm System, based on a Raspberry Pi 4, triggered audible and visual alerts promptly in the event of a fire,
ensuring occupants were promptly warned. Our Smoke Detection System, also utilizing a Raspberry Pi 4, continuously
monitored the environment for smoke using advanced sensors. It served as the primary detection mechanism, activating
the alarm and notification systems upon detecting smoke.

Our Notification System, operating on a Raspberry Pi 4, promptly sent out emergency alerts to predefined contacts,
including local authorities and individuals, ensuring rapid response to the detected fire. We employed a real-time
cloud database to store critical data, facilitating remote monitoring and configuration, thus ensuring optimal system
functionality.

Additionally, we utilized a Pi Pico Device to provide haptic feedback, offering an additional layer of alert through
physical sensation, ensuring even those outside the hearing range of the alarm or with hearing impairments were
alerted.

Our design emphasized scalability, enabling easy expansion with additional sensors or alarm units as needed. Each
component functioned independently, ensuring reliability and ease of maintenance. Communication between components was
facilitated through a local network, with the cloud database supporting remote access and control. We incorporated
advanced communication protocols for efficient data exchange and alerting. The smoke detection system employed the I2C
protocol for sensor communication, while UDP packets facilitated local network communication between the system's
nodes. HTTP requests were utilized for interactions with the cloud-hosted real-time Firebase database, ensuring timely
updates and access to system data.
