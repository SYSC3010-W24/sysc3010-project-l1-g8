\section{Project Description}

\subsection{Motivation}

The motivation for the FANS project was to address the critical problem of preventable fire-related deaths in Canada.
By leveraging modern technological advancements, the FANS system aimed to significantly enhance the effectiveness of
fire alarm systems, reducing response times, and ultimately saving lives. The importance of this project could not be
overstated, as it directly impacts the well-being of Canadians.

\subsection{Problem Statement}

The Fire Alarm Notification System (FANS) was created to address the alarming number of preventable fire deaths in
Canada. There are 220 annual fire-related casualties, and at least one in seven of these deaths occurs in a home
without working smoke alarms \cite{fire-stats}. This emphasizes the shortcomings of traditional fire alarm systems,
which often lacked advanced communication capabilities and robust monitoring capabilities \cite{modern-fire-alarms}.

The FANS project addresses these issues by integrating smoke and temperature sensors with Internet of Things (IoT)
technology. This approach not only covers scenarios where smoke may not reach the detecting device but also offers
real-time notifications via SMS and email, thus notifying homeowners immediately in the event of an emergency.

With configurable thresholds for smoke and temperature alarms and adjustable timeouts that prevent the system from
being unsafely disabled, FANS uses technological advances to significantly improve the effectiveness of fire detection
systems. The ultimate goal of the project is to improve overall fire safety and emergency response, and thereby
mitigate fire disasters and protect the well-being of individuals, families, and communities across Canada
\cite{smoke-alarm-gc}.

\subsection{Overview of Solution}

The FANS system is comprised of several key components in a distributed system for greater efficacy of emergency
detection. The system's architecture is outlined in section \ref{sec:final-solution}.

Our smoke detection system continuously monitors the environment for smoke and temperature changes using different
digital and analog sensors. It serves as the detection mechanism for emergencies, and is responsible for activating the
system's emergency response when abnormal measurements are detected. Emergency activation is done both over local
network communication to other nodes, and over the internet to nodes that may be connected to different networks. This
creates a redundancy for inter-node communication, which can be critical in case of internet connection loss during an
emergency.

The alarm system triggers loud audible and visual alerts in the event of a fire, ensuring occupants are immediately
warned in emergency situations. It can be stationed in high traffic areas for maximum efficacy.

The FANS notification system immediately sends out emergency alerts over email and SMS to a predefined contact list of
users, ensuring rapid response to the detected fire. The system maintains a local database copy of the cloud database
to ensure that these critical messages are still sent if connection to the cloud service is lost.

Finally, FANS utilized a Pi Pico wearable device to provide immediate feedback to individual users, offering an
additional layer of alert through a loud wearable alarm and blinking LEDs for visuals. This is best suited to
individuals with hearing impairments or who may not always be within a close distance to the alarm system nodes.

The entire system can be controlled from a web GUI, where a system administrator can subscribe more contacts to
emergency notifications, modify detection thresholds, timeout the system and watch real-time visualizations of smoke
and temperature measurements being taken.

Our design enables easy expansion with additional sensor or alarm units as needed. Each component functions
independently and has multiple redundancies for inter-node communication, ensuring reliability and ease of maintenance.
