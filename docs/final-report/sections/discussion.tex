\section{Discussion of Final Design}

Although our team adhered to the specifications and objectives originally outlined in our proposal, we did make some
modifications to our original plan as we learned more and got further into our project.

\textbf{Changes Made}

The first change that was made to FANS from the original specification was the LCD screen component. We had originally
planned to add a monitor to the smoke detection unit, but realized during development that it would not have a
practical use since the unit would likely be ceiling mounted where the monitor is not visible.

The second change to our original proposal, which was made several times, was our project timeline. As we developed our
system we realized that our time estimates were not completely accurate. We adjusted our timeline several times to
allocate more resources to development of more difficult portions. For instance, we had to order new components after
the component request period in order to properly interface with our smoke sensor. This caused us to push back the
smoke detection portion of the system until the components arrived.

\textbf{Findings}

The smoke detection unit for our system was not entirely accurate due to electrical noise and some calibration
requirements that could not be met. The data sheet advised to measure air resistance of smoke, but we could not find
any fully reliable values for this online. We simply tuned the sensor to not trigger it's onboard LED in the absence of
smoke. Our system reports a very small PPM measurement of smoke in clear environments due to the calibration issues and
the electrical noise of our breadboard (likely due to the home-made logic level converter).

The temperature sensor was reliable and could accurately detect temperature changes, but had a baseline of around 38
degrees Celsius due to the heat from the CPU accumulating inside the Pi case. Removing the case or mounting the sensor
further away from the CPU could help reduce this.

During our testing, we determined that from moment the smoke detection unit triggered an emergency, the response from
all nodes in the system (alarms, lights, web UI) took a maximum of one second. This is likely dependent on the network
speed, but our system's UDP messages are deliberately short for maximum transmission speed. The longest response time
was for email notifications due to latency in the mail carrier outside of our control, but the notification system and
other local nodes responded to the emergency trigger in under a second.

Scalability was also confirmed by doubling up on specific nodes during testing. We would convert our team's Pis to be
two sensor nodes or two alarm nodes. Since our messages are sent over UDP and internet in a way that can be consumed by
multiple listeners, all of our system nodes could be present in any number without any changes to the system's logic.
