\section{Discussion of Final Design}

Project FANS ( Fire Alarm Notification System) was created to address the urgent issue of preventable fire deaths in
Canada by utilizing modern technologies to improve the effectiveness of fire alarm systems. Our final design
successfully integrates smoke and temperature sensors with Internet of Things (IoT) technology, providing real-time
notifications and improving fire safety measures for individuals and communities across Canada.

Throughout the development of FANS, our team adhered to the specifications and objectives outlined in our initial
proposal and detailed design document. Our goal was to develop a system that would not only detect fire based on smoke
and temperature changes but also immediately notify people of emergencies through a comprehensive and responsive
communication system.

\begin{enumerate}
    \item Complete implementation: We are proud that our project was realized completely and without omissions. Every component, from the smoke detection system to the emergency call system, was implemented as planned. Thank you to our commitment to rigorous planning and effective teamwork, we were able to overcome challenges and ensure that all project objectives were met.
    \item Consistency of protocols: There was no deviation from the originally proposed communication protocols. The use of the I2C protocol for sensor communication, UDP for local network communication and HTTP requests for interaction with the Firebase real-time database were maintained throughout the project. This consistency in the choice of protocols ensured seamless integration of the system components and reliable performance of the FANS.
\end{enumerate}

Our project not only met the expected design specifications, but also performed excellently in the test phases. Here are
some important results and data from our tests:

\begin{itemize}
    \item Sensitivity of the sensors: the smoke and temperature sensors used in the FANS showed high sensitivity and accuracy during testing. They were able to detect slight changes in environmental conditions, ensuring early detection of potential fire hazards. The responsiveness of the sensors is crucial to the effectiveness of the system, as they issue warnings in good time and enable rapid action to be taken.
    \item Efficiency of the notification system: The notification system was tested to ensure that it could immediately trigger an alarm when a fire was detected. The test results showed that the SMS and email notifications were sent without significant delay, usually within seconds of the emergency being detected. This rapid response is crucial for the safety of residents and minimizing potential damage.
    \item Scalability of the system: A key aspect of our design was its scalability, allowing easy integration of additional sensors or alarm units. Tests confirmed that adding new components did not affect the performance of the system, proving the flexibility and adaptability of the design to different environments.
    \item GUI functionality: The Fire Alarm Notification System (FANS) graphical user interface provides an interactive platform for real-time monitoring and control, enabling direct alarm management and seamless integration with the system components and real-time database, significantly improving responsiveness and ease of use.
\end{itemize}

The fire detection system represents a significant advance in fire protection, effectively meeting the need for an
integrated, responsive and reliable fire detection and notification system. Through careful planning, unwavering
commitment to our project goals and meticulous adherence to our design, we have developed a system that will
significantly impact the well-being of Canadians and provide a new level of protection from fire hazards. The successful
completion of this project demonstrates not only our team's technical capabilities but also our commitment to making a
meaningful contribution to public safety.
