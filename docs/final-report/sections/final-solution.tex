\section{Final Solution}

The final solution for FANS is displayed in Figure \ref{fig:deployment}. This solution uses a distributed systems
approach to create a multi-node fire response system controlled via a web interface.

\subsection{Deployment Diagram}

The primary node in the FANS system is the sensor data collection node, which monitors temperature and smoke
concentration levels. All sensor measurements are written to the Firebase cloud database for display by the web UI.

When a measurement is detected above the user-defined threshold, the sensor data collection node sends a UDP message
signifying an emergency over the local network to the notification system and alarm system nodes. Upon receipt of this
message, the notification system will send emails and SMS notifications to all users in its local database. The alarm
system node will begin blinking an LED and sounding an alarm.

In addition to the UDP message, the sensor data collection node will raise an emergency flag in the Firebase database
which signifies to the UI and the wearable haptic alarm that an emergency is active. The haptic alarm wearable device
will then begin blinking an LED and buzzing until the user disables it or the emergency ends.

The UI provides an interface for checking on the status of an emergency and reviewing sensor data in real-time. Users
can disable the emergency response from the UI when the emergency has been resolved, and they can also time out the
system’s emergency response in scenarios where the system may be falsely triggered (smokey cooking). Finally, the UI
also provides an interface for user contact information to be added to the subscriber list for emergency notifications.

\begin{figure}
    \centering
    \includegraphics[width=\imagewidth]{../assets/FANSDeployment.png}
    \caption{The FANS deployment diagram.}
    \label{fig:deployment}
\end{figure}

\subsection{Message Protocol Table}

In the Fire Alarm Notification System (FANS), a variety of communication protocols are meticulously integrated to
ensure seamless interaction among the system components and with the external cloud database, facilitating a robust and
responsive fire alarm solution.

The system’s core, the Smoke Detection System, communicates with its temperature and smoke sensors using the I2C and
SPI protocols over the GPIO pins of a Raspberry Pi 4.

\subsubsection{I2C \& SPI Communication}

The Raspberry Pi 4 utilizes the I2C and SPI protocol to communicate with temperature and smoke sensors, monitoring
environmental conditions to detect potential fire hazards. These messages are outlined in \ref{table:i2c}.

\begin{table}
    \centering
    \begin{tabular}{| c | c | c | c | c |}
        \hline
        Sender         & Receiver               & Message              & Data Format                                          & Protocol                 \\
        \hline
        Raspberry Pi 4 & Temperature Sensor     & \texttt{read\_temp}  & See section 6.2.1 of datasheet \cite{temp-datasheet} & I2C                      \\
        \hline
        Raspberry Pi 4 & Smoke Sensor (via ADC) & \texttt{read\_smoke} & See figure 1.1 of datasheet \cite{adc-datasheet}     & SPI \cite{adc-datasheet} \\
        \hline
    \end{tabular}
    \caption{Messages for I2C communication in FANS.}
    \label{table:i2c}
\end{table}

\subsubsection{Local Area Network Communication}

Nodes within the FANS (smoke detection, alarm, and notification systems) communicate over a local network using UDP
packets, facilitating real-time alerts and system coordination. The messages sent over UDP use numerical value to
encode messages. The representation agreed upon is shown in Table \ref{table:udp-rep}. The messages sent are in Table
\ref{table:udp-messages}

\begin{table}
    \centering
    \begin{tabular}{| c | c |}
        \hline
        Message      & Value \\
        \hline
        Emergency    & 0     \\
        \hline
        No Emergency & 1     \\
        \hline
    \end{tabular}
    \caption{Numerical representation of messages over UDP in FANS.}
    \label{table:udp-rep}
\end{table}

\begin{table}
    \centering
    \begin{tabular}{| c | c | c | c | c |}
        \hline
        Sender                 & Receiver            & Message      & Data Format & Protocol \\
        \hline
        Smoke detection system & Notification system & Emergency    & 0           & UDP      \\
        \hline
        Smoke detection system & Alarm system        & Emergency    & 0           & UDP      \\
        \hline
        Smoke detection system & Notification system & No emergency & 1           & UDP      \\
        \hline
        Smoke detection system & Alarm system        & No Emergency & 1           & UDP      \\
        \hline
    \end{tabular}
    \caption{Messages for local area network communication in FANS.}
    \label{table:udp-messages}
\end{table}

\subsubsection{Cloud Database Communication}

Messages between each node and the Firebase database are shown in Table \ref{table:firebase}.

\begin{table}
    \centering
    \begin{tabular}{| c | c | c | c | c |}
        \hline
        Sender          & Receiver & Message                                    & Data Format                      & Protocol    \\
        \hline
        Smoke Detection & Cloud DB & \texttt{put\_sensor\_data()}               & See listing \ref{lst:sensor}     & HTTP (JSON) \\
        \hline
        GUI             & Cloud DB & \texttt{update\_threshold(new\_threshold)} & See listing \ref{lst:threshold}  & HTTP (JSON) \\
        \hline
        Notification    & Cloud DB & \texttt{query\_contact\_information()}     & See listing \ref{lst:contact}    & HTTP (JSON) \\
        \hline
        Haptic Alarm    & Cloud DB & \texttt{emergency()}                       & See listing \ref{lst:emergency}  & HTTP (JSON) \\
        \hline
        GUI             & Cloud DB & \texttt{get\_sensor\_data()}               & See listing \ref{lst:sensor-get} & HTTP (JSON) \\
        \hline
    \end{tabular}
    \caption{Messages for cloud database communication in FANS.}
    \label{table:firebase}
\end{table}

{
\tiny
\begin{lstlisting}[language=json,label={lst:sensor},caption={Update sensor data message.}]
{
  "method": "PUT",
  "path": "/sensor-data/temperature",
  "headers": {
    "Authorization": "Bearer YOUR_ACCESS_TOKEN",
    "Content-Type": "application/json"
  },
  "body": {
      "data": {
          "temperature": 21.2,
          "timestamp": "2024-03-13T08:37:22"
      }
  }
}
\end{lstlisting}

\begin{lstlisting}[language=json,label={lst:threshold},caption={Threshold update message.}]
{
  "method": "PUT",
  "path": "/system/threshold",
  "headers": {
    "Authorization": "Bearer YOUR_ACCESS_TOKEN",
    "Content-Type": "application/json"
  },
  "body": {
    "newThreshold": 50
  }
}
\end{lstlisting}

\begin{lstlisting}[language=json,label={lst:contact},caption={Request for user contact information.}]
{
  "method": "GET",
  "path": "/user/contact",
  "headers": {
    "Authorization": "Bearer YOUR_ACCESS_TOKEN"
  }
}
\end{lstlisting}

\begin{lstlisting}[language=json,label={lst:emergency},caption={Request for emergency flag.}]
{
  "method": "GET",
  "path": "/system/emergency",
  "headers": {
    "Authorization": "Bearer YOUR_ACCESS_TOKEN"
  }
}
\end{lstlisting}

\begin{lstlisting}[language=json,label={lst:sensor-get},caption={Request for latest sensor data.}]
{
  "method": "GET",
  "path": "/sensor-data",
  "headers": {
    "Authorization": "Bearer YOUR_ACCESS_TOKEN"
  }
}
\end{lstlisting}
}

\subsubsection{User Notification Communication}

The notification system communicates with users through email, employing standard internet protocols to ensure timely
and effective alerts.

\begin{table}[H]
    \centering
    \begin{tabular}{| c | c | c | c | c |}
        \hline
        Sender              & Receiver   & Message                & Data Format                 & Protocol     \\
        \hline
        Notification System & User Inbox & Emergency notification & See listing \ref{lst:email} & SMTP (Email) \\
        \hline
    \end{tabular}
    \caption{Messages for user notification communication in FANS.}
\end{table}

{
\tiny
\begin{lstlisting}[label={lst:email},caption={Email notification for detected emergency in FANS.}]
FROM: notification@example.com
TO: user@example.com
SUBJECT: Fire Alarm Notification

Dear [User's Name],

This is an emergency notification. Please exit the building.

Emergency detected: [Date, Time]

Stay safe!
\end{lstlisting}
}

\subsection{Sequence Diagrams}

The FANS system will satisfy six core use cases highlighted and illustrated in the message sequence diagrams below.

\subsubsection{Trigger Emergency Use Case}

The trigger emergency use case shown in Figure \ref{fig:trigger-emerg} represents the critical functionality of the
FANS, where the system detects a potential fire through its smoke detection system and initiates a series of automated
responses to mitigate the situation. As depicted in the message sequence diagram, the process begins when the smoke
detection system identifies smoke and potentially high temperatures indicative of a fire. This detection triggers the
system to send a notification to the alarm system and the notification system. The alarm system responds by sounding an
audible and visible alert to notify occupants of the building immediately. Concurrently, the notification system sends
out emergency alerts via SMS and email to all registered users, ensuring they are informed of the danger regardless of
their current location. This response ensures that all the users are promptly alerted to the emergency, resulting in a
quick evacuation and response to the detected threat.

\begin{figure}
    \centering
    \includegraphics[width=\imagewidth]{../assets/sequence/TriggerEmergencyUseCaseSequenceDiagram.png}
    \caption{The sequence diagram for the trigger emergency use case.}
    \label{fig:trigger-emerg}
\end{figure}

\subsubsection{Add New Contact Information Use Case}

This use case shown in Figure \ref{fig:add-contact} outlined the process by which users added new contact information
to the FANS database. Through the web GUI, users entered new contact details such as phone numbers and email addresses,
enabling the system to send emergency notifications. Upon submission, these details were successfully updated in the
real-time database. The message sequence diagram for this use case (not shown) illustrated the interactions between the
user, the web GUI, and the database, culminating in the notification system being updated with the new contact
information. This ensured that the FANS could reach the user through various channels in the event of an emergency,
thereby enhancing the system’s effectiveness in communicating critical alerts.

\begin{figure}
    \centering
    \includegraphics[width=\imagewidth]{../assets/sequence/AddingNewContactInformationSequenceDiagram.png}
    \caption{The sequence diagram for the adding new contact information use case.}
    \label{fig:add-contact}
\end{figure}

\subsubsection{Change Emergency Threshold Use Case}

Adjusting the smoke detection threshold was an essential feature that allowed users to customize the sensitivity of the
FANS based on environmental conditions and personal preferences. This use case shown in \ref{fig:change-thresh}
involved a user accessing the web GUI to modify the threshold settings determining when the smoke detection system
should trigger an alert. The message sequence diagram (not shown) visualized the steps involved, from the user’s
interaction with the GUI to update the threshold to the real-time database’s role in storing this new setting. The
smoke detection system then retrieved and applied the updated threshold, ensuring that the FANS operated according to
the user’s specifications, balancing sensitivity and minimizing false alarms.

\begin{figure}
    \centering
    \includegraphics[width=\imagewidth]{../assets/sequence/ChangingSmokeDetectionThresholdSequenceDiagram.png}
    \caption{The sequence diagram for the changing emergency threshold use case.}
    \label{fig:change-thresh}
\end{figure}

\subsubsection{Emergency Response Use Case}

The scenario in Figure \ref{fig:emergency-resp} highlights the FANS’s capability to notify users in the event of a
detected fire. Upon detecting a fire, the notification system successfully queried the cloud database for contact
information and then sent out alerts via SMS and email to all users. The message sequence diagram for this use case
(provided above) captured the sequence of these interactions, showcasing how the system ensured that occupants were
informed and ready to address the emergency promptly.

\begin{figure}
    \centering
    \includegraphics[width=\imagewidth]{../assets/sequence/EmergencyResponseUseCase.png}
    \caption{The sequence diagram for the emergency response use case.}
    \label{fig:emergency-resp}
\end{figure}
