\section{Final Solution}

\subsection{Deployment Diagram}

\subsection{Message Protocol Table}

In the Fire Alarm Notification System (FANS), a variety of communication protocols are meticulously integrated to ensure
seamless interaction among the system components and with the external cloud database, facilitating a robust and
responsive fire alarm solution.

The system’s core, the Smoke Detection System, communicates with its temperature and smoke sensors using the I2C and SPI
protocols over the GPIO pins of a Raspberry Pi 4.

\subsubsection{I2C Communication}

The Raspberry Pi 4 utilizes the I2C protocol to communicate with an array of temperature and smoke sensors, monitoring
environmental conditions to detect potential fire hazards. This is not visible in the sequence diagrams as they show
the bigger picture of the communication of the nodes with the sensors.

% TODO: fix that the smoke sensor uses I2C
\begin{table}[H]
    \centering
    \begin{tabular}{| c | c | c | c | c |}
        \hline
        Sender         & Receiver               & Message              & Data Format                                          & Protocol                 \\
        \hline
        Raspberry Pi 4 & Temperature Sensor     & \texttt{read\_temp}  & See section 6.2.1 of datasheet \cite{temp-datasheet} & I2C                      \\
        \hline
        Raspberry Pi 4 & Smoke Sensor (via ADC) & \texttt{read\_smoke} & See figure 1.1 of datasheet \cite{adc-datasheet}     & SPI \cite{adc-datasheet} \\
        \hline
    \end{tabular}
    \caption{Messages for I2C communication in FANS.}
\end{table}

\subsubsection{Local Area Network Communication}

Nodes within the FANS (smoke detection, alarm, and notification systems) communicate over a local network using UDP
packets, facilitating real-time alerts and system coordination.

The messages sent over UDP use numerical value to encode messages. The representation agreed upon is as follows:

\begin{table}[H]
    \centering
    \begin{tabular}{| c | c |}
        \hline
        Message      & Value \\
        \hline
        Emergency    & 0     \\
        \hline
        No Emergency & 1     \\
        \hline
    \end{tabular}
    \caption{Numerical representation of messages over UDP in FANS.}
\end{table}

\begin{table}[H]
    \centering
    \begin{tabular}{| c | c | c | c | c |}
        \hline
        Sender                 & Receiver            & Message      & Data Format & Protocol \\
        \hline
        Smoke detection system & Notification system & Emergency    & 0           & UDP      \\
        \hline
        Smoke detection system & Alarm system        & Emergency    & 0           & UDP      \\
        \hline
        Smoke detection system & Notification system & No emergency & 1           & UDP      \\
        \hline
        Smoke detection system & Alarm system        & No Emergency & 1           & UDP      \\
        \hline
    \end{tabular}
    \caption{Messages for local area network communication in FANS.}
\end{table}

\subsubsection{Cloud Database Communication}

\begin{table}[H]
    \centering
    \begin{tabular}{| c | c | c | c | c |}
        \hline
        Sender          & Receiver & Message                                    & Data Format                      & Protocol    \\
        \hline
        Smoke Detection & Cloud DB & \texttt{put\_sensor\_data()}               & See listing \ref{lst:sensor}     & HTTP (JSON) \\
        \hline
        GUI             & Cloud DB & \texttt{update\_threshold(new\_threshold)} & See listing \ref{lst:threshold}  & HTTP (JSON) \\
        \hline
        Notification    & Cloud DB & \texttt{query\_contact\_information()}     & See listing \ref{lst:contact}    & HTTP (JSON) \\
        \hline
        Haptic Alarm    & Cloud DB & \texttt{emergency()}                       & See listing \ref{lst:emergency}  & HTTP (JSON) \\
        \hline
        GUI             & Cloud DB & \texttt{get\_sensor\_data()}               & See listing \ref{lst:sensor-get} & HTTP (JSON) \\
        \hline
    \end{tabular}
    \caption{Messages for cloud database communication in FANS.}
\end{table}

\begin{lstlisting}[language=json,label={lst:sensor},caption={Update sensor data message.}]
{
  "method": "PUT",
  "path": "/sensor-data/temperature",
  "headers": {
    "Authorization": "Bearer YOUR_ACCESS_TOKEN",
    "Content-Type": "application/json"
  },
  "body": {
      "data": {
          "temperature": 21.2,
          "timestamp": "2024-03-13T08:37:22"
      }
  }
}
\end{lstlisting}

\begin{lstlisting}[language=json,label={lst:threshold},caption={Threshold update message.}]
{
  "method": "PUT",
  "path": "/system/threshold",
  "headers": {
    "Authorization": "Bearer YOUR_ACCESS_TOKEN",
    "Content-Type": "application/json"
  },
  "body": {
    "newThreshold": 50
  }
}
\end{lstlisting}

\begin{lstlisting}[language=json,label={lst:contact},caption={Request for user contact information.}]
{
  "method": "GET",
  "path": "/user/contact",
  "headers": {
    "Authorization": "Bearer YOUR_ACCESS_TOKEN"
  }
}
\end{lstlisting}

\begin{lstlisting}[language=json,label={lst:emergency},caption={Request for emergency flag.}]
{
  "method": "GET",
  "path": "/system/emergency",
  "headers": {
    "Authorization": "Bearer YOUR_ACCESS_TOKEN"
  }
}
\end{lstlisting}

\begin{lstlisting}[language=json,label={lst:sensor-get},caption={Request for latest sensor data.}]
{
  "method": "GET",
  "path": "/sensor-data",
  "headers": {
    "Authorization": "Bearer YOUR_ACCESS_TOKEN"
  }
}
\end{lstlisting}

\subsubsection{User Notification Communication}

The notification system communicates with users through email, employing standard internet protocols to ensure timely
and effective alerts.

\begin{table}[H]
    \centering
    \begin{tabular}{| c | c | c | c | c |}
        \hline
        Sender              & Receiver   & Message                & Data Format                 & Protocol     \\
        \hline
        Notification System & User Inbox & Emergency notification & See listing \ref{lst:email} & SMTP (Email) \\
        \hline
    \end{tabular}
    \caption{Messages for user notification communication in FANS.}
\end{table}

\begin{lstlisting}[label={lst:email},caption={Email notification for detected emergency in FANS.}]
FROM: notification@example.com
TO: user@example.com
SUBJECT: Fire Alarm Notification

Dear [User's Name],

This is an emergency notification. Please exit the building.

Emergency detected: [Date, Time]

Stay safe!
\end{lstlisting}

\subsection{Sequence Diagrams}
