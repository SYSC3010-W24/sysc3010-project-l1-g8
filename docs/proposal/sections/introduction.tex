\section{Introduction}

This document presents the proposal for Fire Alarm Notification System (FANS), a comprehensive solution addressing the
alarming rate of preventable fire deaths in Canada. \cite{smoke-alarm-gc} The L1-G8 group has developed FANS as a
multi-device fire alarm system with advanced features to significantly reduce fire deaths through timely detection and
response to fire emergencies.

The proposal reveals the detailed system design and comprehensive work plan for the development and implementation of
FANS to reduce the number of preventable fire-related deaths in Canada. It outlines the system's architecture and
provides a structured work plan detailing the phases of development, testing and deployment to ensure the timely
release of FANS to effectively mitigate fire disasters.

\subsection{Background}

Every year in Canada, 220 people die from fire-related causes. \cite{fire-stats} At least 1 in 7 of these deaths
occurred in homes without a working smoke alarm. \cite{fire-stats} This is an alarming number of deaths, which FANS
aims to prevent.

As a result of these preventable deaths, the government of Canada has released guidelines for smoke alarm installation
and care. \cite{smoke-alarm-gc} FANS aims to take a modern approach to fire safety by integrating its smoke alarm
system with the Internet of Things (IoT).

Traditional fire alarm systems often lack modern communication features and real-time monitoring capabilities. These
limitations contribute to delays in response time and can lead to preventable fire-related deaths. Modern systems make
use of not only smoke detection sensors, but also heat detectors that can detect a rise in temperature due to fires.
\cite{modern-fire-alarms} Smoke detection systems are also subject to unsafe disarming by users who have their alarm
installed close to their kitchen space. False alarms can be frustrating while cooking, but disabling the smoke alarm is
a dangerous action that puts the user at risk of forgetting to re-enable it. \cite{disabling-alarm}

The FANS project aims to address the shortcomings of traditional systems and implement the strengths of modern systems
by introducing a comprehensive multi-system solution. FANS will use both smoke and temperature sensors to cover
situations where smoke cannot reach the detecting device. It will issue real-time notifications via SMS and email to
notify homeowners of an emergency. Finally, FANS will offer configurable thresholds for smoke and temperature alarms, as
well as configurable timeouts for the system to prevent disgruntled users from unsafely disabling their alarm system
while cooking.

By leveraging technological advancements, FANS seeks to improve the effectiveness of fire alarm systems and reduce
fire-related fatalities in Canada.

\subsection{Motivation}

The motivation for the FANS project is to address the critical problem of preventable fire-related deaths in Canada.
Fire safety is a fundamental concern for individuals, families, and communities nationwide. By leveraging modern
technological advancements, the FANS system aims to significantly enhance the effectiveness of fire alarm systems,
reducing response times, and ultimately saving lives. The importance of this project cannot be overstated, as it
directly impacts the safety and well-being of Canadians.

\subsection{Project Objective}

The overarching objective of the FANS project is to develop and deploy a functional fire detection and notification
system that enhances fire safety. FANS accomplishes this by providing real-time fire detection, timely notifications to
users, and proactive monitoring of environmental conditions.

\subsection{Specific Goals}

The goals that the FANS development team aims to achieve in this project are as follows:

\begin{enumerate}
    \item Develop and test a smoke detection algorithm on Raspberry Pi to ensure accurate and reliable fire detection.

    \item Integrate various sensors with Raspberry Pi for real-time data collection, including temperature and smoke levels.

    \item Implement a robust notification system capable of triggering alarms at installation sites and sending email alerts to
          users in affected areas.

    \item Create a user-friendly web interface for monitoring and interaction, allowing users to access real-time data, configure
          system settings, and receive alerts.

    \item Establish a secure and efficient database for storing employee contact information and system data, ensuring the
          availability of necessary information for emergency response.
\end{enumerate}

\subsubsection{Functional Requirements}

\begin{enumerate}
    \item Consistent accuracy from the smoke sensors to avoid any false alarms/triggers.

    \item Real-time data collection and monitoring of temperature and smoke levels.

    \item Timely and reliable email notifications to users during fire emergencies.

    \item User-friendly web interface with intuitive design and real-time updates.

    \item Database with secure storage and retrieval of employee contact information and system data.
\end{enumerate}

\subsubsection{Non-Functional Requirements}

\begin{enumerate}
    \item High system reliability and availability to ensure immediate response during emergencies.

    \item Scalability to accommodate a growing number of users and installation sites.

    \item Data security and privacy measures to protect sensitive information.
\end{enumerate}
