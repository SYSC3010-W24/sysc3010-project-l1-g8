\section{Problem Statement}

The need for the Fire Alarm Notification System (FANS) stems from the alarming number of preventable fire deaths in
Canada, where 220 people die in fires each year, with at least one in seven of these deaths occurring in homes without
working smoke alarms \cite{fire-stats}. This critical problem underscores the shortcomings of traditional fire alarm
systems, which often lack advanced communication capabilities and real-time monitoring capabilities
\cite{modern-fire-alarms}. These limitations not only contribute to delayed response times, but also to preventable
deaths, underscoring the urgent need for an advanced fire alarm solution.

Traditional systems' shortcomings, coupled with the fact that smoke detection systems are sometimes unsafely disarmed
by users to avoid false alarms—especially those installed close to kitchen spaces—further exacerbate the problem. The
FANS project seeks to address these issues by integrating smoke and temperature sensors with Internet of Things (IoT)
technology. This approach not only aims to cover scenarios where smoke may not reach the detecting device but also
offers real-time notifications via SMS and email, thus notifying homeowners immediately in the event of an emergency.

By providing configurable thresholds for smoke and temperature alarms and adjustable timeouts that prevent the system
from being deactivated in an unsafe manner, FANS aims to use technological advances to significantly improve the
effectiveness of fire detection systems. The ultimate goal of the project is to reduce response times, improve overall
fire safety and thereby mitigate fire disasters and protect the well-being of individuals, families and communities
across Canada \cite{smoke-alarm-gc}.

\subsection{Functional Requirements}

\begin{enumerate}
    \item \textbf{Accurate and Reliable Detection} \\ Implement dual-sensor technology combining photoelectric and thermistor-based
          sensors for accurate smoke and temperature detection, alongside unit testing routines to ensure reliability.

    \item \textbf{Real-time Monitoring and Alerts} \\ Utilize a cloud-based platform for continuous real-time data
          storage, with a reliable communication channel for immediate alert dispatch. The database must be able to
          serve the stored data to multiple clients over an internet connection as it is updated in real-time.

    \item \textbf{User-configurable Sensitivity} \\ Provide an interface for users to customize their alarm
          system sensitivity preferences, including both temperature and smoke detection trigger thresholds.

    \item \textbf{User-defined System Timeout} \\ Provide an interface for users to temporarily disable their alarm
          systems for a pre-determined length of time, allowing the user to avoid triggering emergencies in high-smoke
          environments (such as kitchens).

    \item \textbf{Comprehensive Notification System} \\ Initiate on-site alarms and send email to users and designated contacts to ensure
          widespread awareness during emergencies.

    \item \textbf{Wearable Emergency Response} \\ Incorporate a haptic alarm wearable device that will notify users
          through vibration in active emergency situations. This is ideal for situations where auditory alarms may not
          be effective or where users with hearing impairments are at risk of a fire hazard, ensuring maximum reach and
          effectiveness of alerts.

\end{enumerate}

By adopting these key requirements, FANS aims to deliver a technologically advanced, user-friendly system for timely
fire detection and response, aiming to reduce fire-related fatalities and enhance safety for individuals and
communities across Canada.
