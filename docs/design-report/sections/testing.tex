\section{Test Plans}

Testing is an important part of any system's design. It ensures that project iterations follow the specification,
behave appropriately and are less likely to include bugs.

\subsection{End-to-end Communication Demo}

The end-to-end communication demo should demonstrate all of the major forms of communication present in the FANS
system. In order to achieve this level of demonstration, the following communication examples will be showcased:

\begin{itemize}
    \item Sensor data upload to Firebase over internet.
    \item Email notifications to affected users over internet (email protocol).
    \item Haptic alarm buzzing in accordance with emergency flag in Firebase over internet.
    \item Display of sensor data on the user interface over internet.
    \item Emergency signal from sensor data collection system to the alarm system on local network.
\end{itemize}

It should be noted that the test for the sensor data upload over internet also demonstrates I2C communication between
the Pi 4 and the SenseHat board.

First is the uploading of sensor data to the Firebase database over the internet. For this communication sequence, the
sensor data collection system will be run on one of the Raspberry Pis. It will read temperature data from the SenseHat
and post this data to the Firebase database using Pyrebase. The demonstration can be verified by watching the Firebase
database console, since the data will be updated in real-time.

Second will be the email notifications to affected users over the internet using email protocol. To achieve this
demonstration, the email notification system will be run on one of the Raspberry Pis. Using Pytest, the program will
first send an email from the FANS email back to itself. Then, the program will use \texttt{imaplib} to login to the
FANS email account and verify that the latest message in its inbox contains the same email message that was sent in the
test. If the messages match, Pytest will report a success. This test is fully automated.

Next is the haptic alarm buzzing. For this test, the haptic alarm program will be run on one of the Raspberry Pi's to
emulate the Pi Pico (as it has not arrived at the time of writing). When the program is run, it will poll the Firebase
database using \texttt{pyrebase} to check if the emergency flag has been raised. The flag can either be raised by hand
(the tester modifies the flag using the Firebase console) or it can be set by running the sensor data collection system
with a low emergency reporting threshold for temperature. In either case, when the flag is raised, the output of the
haptic alarm can be verified by observing the console messages. Console printing will be used in place of the buzzer
actuator while we wait for components. A console message of "buzz" would indicate the buzzer buzzing, and no console
messages indicate no emergency while the alarm is polling the database.

To test the display of sensor data collection on the user interface, the GUI will be run on one of the Raspberry Pis
using Flask. When a user clicks the "refresh" button, the Flask API endpoint responsible for returning temperature data
will be hit. The endpoint will use \texttt{pyrebase} to request the latest temperature data from Firebase and then will
respond with JSON. The GUI's JavaScript logic will simply print this JSON as text content of an HTML paragraph tag to
show that the connection is present.

Finally, to demonstrate UDP communication over the local area network, both the sensor data collection system and the
alarm system will be run on two separate Pis. Both programs will first upload their public IP to the Firebase database
under a 'devices' table, and then read the other device's IP. The sensor data collection system should have a
sufficiently low emergency threshold to trigger an emergency at room temperature (only for testing purposes). Once the
sensor data collection system detects an emergency, it will send a UDP message containing the numerical data '0' (the
agreed upon encoding for signifying an emergency) over the LAN addressed to the alarm system. The alarm system will
receive the emergency notification, and instead of activating an actuator it will print "Received 0" to the console.
This can be manually verified.

\subsection{Unit Test Demo}

The purpose of the unit test demo is to demonstrate that the individual components of our system behave as expected and
do not contain logical or physical errors. This will sanity check our hardware components and our software logic.

\subsubsection{Hardware}

The hardware components that will need to be tested are:

\begin{itemize}
    \item Smoke sensor
    \item Temperature sensor
    \item Piezoelectric buzzer
    \item LCD screen
\end{itemize}

\textbf{Smoke Detector}

The smoke detector will be difficult to test due to the nature of what it measures. It will not be possible to light a
fire within the school or near campus because that will set of existing fire suppression systems and may be dangerous.
Testing the smoke detector in smokey environments can be done by:

\begin{itemize}
    \item Running the data collection from the smoke detector in a smoking zone on campus.
    \item Running the data collection from the smoke detector off campus and lighting a fire nearby.
\end{itemize}

For both of these tests, it should be verified that smoke levels are detected. This will be a one-time sanity test of
the sensor. The smoke detector can also be tested in smokeless environments by collecting data from the sensor indoors
on campus and ensuring that no smoke is detected.

\textbf{Temperature Sensor}

To test the temperature sensor, we can collect data from the sensor in the lab setting and ensure that measurements are
within the acceptable range for room temperature.

\textbf{Piezoelectric Buzzer}

To test the piezoelectric buzzer, we can test turning on and off the buzzer and verifying that it produces noise.
Additionally, we can set the buzzer to produce different frequencies and verify that they are set correctly by checking
with a instrument tuning app.

\textbf{LCD Screen}

Finally, to test the LCD screen, several test messages can be displayed on screen and visually inspected for
correctness as a sanity check.

\subsubsection{Database Integrity}

To test cloud database integrity, we can perform several test and security checks.

\textbf{Insecure Access}

To test the security of the cloud database, we can attempt to perform read and write operations using an incorrect API
key and ensure that the requests are denied.

\textbf{Stress Test}

To stress test the cloud database, several instances of the sensor data collection program can be run on multiple
Raspberry Pis, all of which will upload sensor data to the database. The database should:

\begin{itemize}
    \item Keep up with all of the simultaneous requests.
    \item Avoid any overwriting/race conditions with concurrent write operations.
\end{itemize}

\subsubsection{Software}

% TODO: Write unit test plans for the software

\subsection{Final Demo}

% TODO: Write final demo
