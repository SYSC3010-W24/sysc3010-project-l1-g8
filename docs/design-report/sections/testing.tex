\section{Test Plans}

Testing is an important part of any system's design. It ensures that project iterations follow the specification,
behave appropriately and are less likely to include bugs.

\subsection{End-to-end Communication Demo}

The end-to-end communication demo should demonstrate all of the major forms of communication present in the FANS
system. In order to achieve this level of demonstration, the following communication examples will be showcased:

\begin{itemize}
    \item Sensor data upload to Firebase over internet.
    \item Email notifications to affected users over internet (email protocol).
    \item Haptic alarm buzzing in accordance with emergency flag in Firebase over internet.
    \item Display of sensor data on the user interface over internet.
    \item Emergency signal from sensor data collection system to the alarm system on local network.
\end{itemize}

It should be noted that the test for the sensor data upload over internet also demonstrates I2C communication between
the Pi 4 and the SenseHat board.

First is the uploading of sensor data to the Firebase database over the internet. For this communication sequence, the
sensor data collection system will be run on one of the Raspberry Pis. It will read temperature data from the SenseHat
and post this data to the Firebase database using Pyrebase. The demonstration can be verified by watching the Firebase
database console, since the data will be updated in real-time.

Second will be the email notifications to affected users over the internet using email protocol. To achieve this
demonstration, the email notification system will be run on one of the Raspberry Pis. Using Pytest, the program will
first send an email from the FANS email back to itself. Then, the program will use \texttt{imaplib} to login to the
FANS email account and verify that the latest message in its inbox contains the same email message that was sent in the
test. If the messages match, Pytest will report a success. This test is fully automated.

Next is the haptic alarm buzzing. For this test, the haptic alarm program will be run on one of the Raspberry Pi's to
emulate the Pi Pico (as it has not arrived at the time of writing). When the program is run, it will poll the Firebase
database using \texttt{pyrebase} to check if the emergency flag has been raised. The flag can either be raised by hand
(the tester modifies the flag using the Firebase console) or it can be set by running the sensor data collection system
with a low emergency reporting threshold for temperature. In either case, when the flag is raised, the output of the
haptic alarm can be verified by observing the console messages. Console printing will be used in place of the buzzer
actuator while we wait for components. A console message of "buzz" would indicate the buzzer buzzing, and no console
messages indicate no emergency while the alarm is polling the database.

To test the display of sensor data collection on the user interface, the GUI will be run on one of the Raspberry Pis
using Flask. When a user clicks the "refresh" button, the Flask API endpoint responsible for returning temperature data
will be hit. The endpoint will use \texttt{pyrebase} to request the latest temperature data from Firebase and then will
respond with JSON. The GUI's JavaScript logic will simply print this JSON as text content of an HTML paragraph tag to
show that the connection is present.

Finally, to demonstrate UDP communication over the local area network, both the sensor data collection system and the
alarm system will be run on two separate Pis. Both programs will first upload their public IP to the Firebase database
under a 'devices' table, and then read the other device's IP. The sensor data collection system should have a
sufficiently low emergency threshold to trigger an emergency at room temperature (only for testing purposes). Once the
sensor data collection system detects an emergency, it will send a UDP message containing the numerical data '0' (the
agreed upon encoding for signifying an emergency) over the LAN addressed to the alarm system. The alarm system will
receive the emergency notification, and instead of activating an actuator it will print "Received 0" to the console.
This can be manually verified.

\subsection{Unit Test Demo}

\subsection{Final Demo}
